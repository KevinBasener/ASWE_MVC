%!TEX root = ../main.tex

\pagestyle{empty}

% override abstract headline
\renewcommand{\abstractname}{Abstract}

\begin{abstract}
Diese Seminararbeit untersucht das Model-View-Controller (MVC) Architektur-Pattern und vergleicht es mit verwandten Architektur-Pattern wie Model-View-Presenter (MVP) und Model-View-ViewModel (MVVM). Ziel ist es, die Vor- und Nachteile dieser Architektur-Pattern hinsichtlich Modularität, Kohäsion, Kopplung und Code Overhead zu bewerten. Die Arbeit beginnt mit einer detaillierten Vorstellung der MVC-Architektur, einschließlich ihrer Eigenschaften, grafischen Darstellung und eines Codebeispiels. Anschließend werden die Bewertungsmethoden für Architektur-Pattern erläutert. Die Untersuchung zeigt, dass jedes Architektur-Pattern spezifische Stärken und Schwächen aufweist, die je nach Anwendungsszenario unterschiedlich gewichtet werden können.
\end{abstract}