%!TEX root = ../main.tex

\pagestyle{empty}

% override abstract headline
\renewcommand{\abstractname}{Abstract}

\begin{abstract}
Diese Seminararbeit untersucht das Model-View-Controller (MVC) Architekturmuster und vergleicht es mit verwandten Mustern wie Model-View-Presenter (MVP) und Model-View-ViewModel (MVVM). Ziel ist es, die Vor- und Nachteile dieser Muster hinsichtlich Modularität, Kohäsion, Kopplung und Code Overhead zu bewerten. Die Arbeit beginnt mit einer detaillierten Vorstellung der MVC-Architektur, einschließlich ihrer Eigenschaften, grafischen Darstellung und eines Codebeispiels. Anschließend werden die Bewertungsmethoden für Design Patterns erläutert. Die Untersuchung zeigt, dass jedes Muster spezifische Stärken und Schwächen aufweist, die je nach Anwendungsszenario unterschiedlich gewichtet werden können.
\end{abstract}