%!TEX root = ../../main.tex

\chapter{Einleitung}

In der Softwareentwicklung spielen Architektur-Pattern eine entscheidende Rolle, um die Struktur und Wartbarkeit von Anwendungen zu verbessern. Eines der bekanntesten und am häufigsten verwendeten Pattern ist das \ac{MVC} Pattern. Dieses Architektur-Pattern teilt eine Anwendung in drei Hauptkomponenten: Model, View und Controller. Jede dieser Komponenten hat spezifische Aufgaben und Verantwortlichkeiten, was zu einer klaren Trennung der Anliegen führt und die Wartung und Erweiterung der Anwendung erleichtert.

Neben \ac{MVC} gibt es weitere verwandte Architektur-Pattern wie \ac{MVP} und \ac{MVVM}, die ähnliche Ziele verfolgen, aber unterschiedliche Ansätze zur Trennung von Logik und Darstellung bieten. Diese Pattern sind besonders nützlich in komplexen Anwendungen, bei denen eine klare Strukturierung und Modularität entscheidend sind.

In dieser Arbeit werden die Eigenschaften und Vorzüge des \ac{MVC}-Patterns sowie die verwandten Architektur-Pattern \ac{MVP} und \ac{MVVM} untersucht. Dabei wird auf die Modularität, Kohäsion, Kopplung und den Code Overhead eingegangen, um die jeweiligen Vor- und Nachteile zu ermitteln und eine objektive Analyse zu gewährleisten.
