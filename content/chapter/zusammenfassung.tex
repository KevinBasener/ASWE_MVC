%!TEX root = ../../main.tex

\chapter{Zusammenfassung}

Die Untersuchung der Architektur-Pattern \ac{MVC}, \ac{MVP} und \ac{MVVM} zeigt, dass jedes dieser Pattern spezifische Vor- und Nachteile bietet, die je nach Anwendungsfall unterschiedlich gewichtet werden können. Das \ac{MVC}-Pattern zeichnet sich durch eine klare Trennung der Verantwortlichkeiten aus, was die Wartbarkeit und Erweiterbarkeit der Anwendung erleichtert. Allerdings kann die starke Kopplung zwischen Controller und View zu Herausforderungen bei der Flexibilität und Testbarkeit führen.

Das \ac{MVP}-Pattern bietet eine verbesserte Kontrolle über die Interaktion zwischen den Komponenten und fördert eine hohe Modularität und Kohäsion. Die enge Kopplung zwischen View und Presenter kann jedoch die Testbarkeit beeinträchtigen, und der zusätzliche Code Overhead kann den Entwicklungsaufwand erhöhen.

Das \ac{MVVM}-Pattern ermöglicht eine automatische Synchronisierung zwischen View und ViewModel, was die Wartbarkeit und Wiederverwendbarkeit der Komponenten verbessert. Die potenziell hohe Kopplung zwischen View und ViewModel sowie der zusätzliche Code Overhead durch die Implementierung von Bindungen können jedoch zu Herausforderungen führen.

Insgesamt bieten alle drei Architektur-Pattern wertvolle Ansätze zur Strukturierung von Softwareanwendungen. Die Wahl des geeigneten Pattern hängt von den spezifischen Anforderungen und Zielen des Projekts ab. Durch die Analyse der Modularität, Kohäsion, Kopplung und des Code Overheads können Entwickler fundierte Entscheidungen treffen, um die bestmögliche Architektur für ihre Anwendungen zu wählen.
