\chapter{Vor- und Nachteile des MVCs}

In der Arbeit \textit{A Journey Through the Land of 
Model-View Design Patterns} /cite{aihara2012mvc} werden verschiedene 
Entwurfsmuster wie MVC, MVVM und MVP untersucht 
und miteinander verglichen. Dabei werden ihre Vor- und 
Nachteile analysiert.

Das Model-View-Controller (MVC) Muster
sorgt für eine klare Trennung der 
Verantwortlichkeiten zwischen dem Modell 
(Domänenlogik), der View (UI-Präsentation) und dem 
Controller (Verarbeitung von Benutzereingaben). Ein 
Hauptnachteil von MVC ist jedoch die begrenzte 
Fähigkeit, den Status der View zu verwalten, da 
dieser nicht natürlich Teil des Domänenmodells ist. 

Außerdem setzt MVC oft eine strikte Entkopplung von 
View und Controller voraus, was in der Praxis nicht 
immer umsetzbar ist, insbesondere wenn Frameworks 
diese Aufgaben von Natur aus kombinieren 
\cite{aihara2012mvc}.

Das Model-View-ViewModel (MVVM) verbessert MVC, 
indem es eine ViewModel-Schicht einführt, die den 
View-Status separat vom Domänenmodell verwaltet. 
Dieses Muster ist besonders nützlich in Szenarien, 
in denen mehrere Views die gleichen Daten 
darstellen müssen. Ein entscheidender Vorteil ist 
die deklarative Spezifikation und die automatisierte 
Synchronisierung zwischen View und ViewModel, was 
das UI-Management vereinfacht.

Allerdings kann MVVMs Abhängigkeit von der 
Observer-Synchronisation zu Leistungsproblemen 
führen, insbesondere bei groß angelegten 
Anwendungen oder mehreren Views \cite{qureshi2013comparison}.

Model-View-Presenter (MVP) bietet mehr Flexibilität, 
indem es mehr Kontrolle darüber gibt, wie die 
Verantwortlichkeiten auf die Komponenten verteilt 
werden. Im Gegensatz zu MVC erzwingt MVP keine 
strikte Entkopplung der Funktionen von View und 
Controller, was jedoch die Code-Komplexität erhöhen 
und die Wartbarkeit in manchen Fällen verringern 
kann.

Diese Flexibilität macht es jedoch zu einem 
geeigneten Muster für eine Vielzahl von 
Anwendungsszenarien, insbesondere dort, wo die 
Unterstützung für Datenbindungen begrenzt ist 
\cite{aihara2010usability}.

Insgesamt bietet jedes dieser Muster spezifische 
Vorteile, aber die Wahl zwischen ihnen hängt stark 
vom Anwendungsfall und der zugrunde liegenden 
Technologie ab. MVVM ist beispielsweise besonders 
geeignet für Anwendungen mit starker 
Datenbindung, während MVP eine manuellere 
Steuerung der Synchronisierung ermöglicht und so 
besser für komplexe Anwendungsfälle anpassbar ist.
