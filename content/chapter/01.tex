%!TEX root = ../../main.tex

\chapter{Methodik}
\label{chap:methodik}

\section{Kriterien zur Bewertung der MVC-Architektur}

Olsson et al. \cite{olsson2015evolution} legen Kriterien fest, um 
MVC in Videospielen zu untersuchen.
Die Bewertung der Vor- und Nachteile der MVC-Architektur erfolgt orientiert 
sich anhand der Kriterien, die zur Analyse der Architekturstruktur und ihrer 
Effektivität herangezogen werden in \cite{olsson2015evolution}. Im Fokus stehen 
die folgenden Aspekte:

\begin{itemize}
    \item \textbf{Modularität}: Untersucht wird, wie klar die Trennung 
    von Modell, View und Controller ist. Ein hohes Maß an Trennung 
    deutet auf eine verbesserte Wartbarkeit hin.
    
    \item \textbf{Kohäsion und Kopplung}: Die Kohäsion misst, wie 
    eng zusammenhängend die Verantwortlichkeiten der einzelnen 
    Komponenten sind. Niedrige Kopplung zwischen Klassen ist ein 
    positives Zeichen für Wiederverwendbarkeit.
    
    \item \textbf{Code Overhead}: Hierbei wird analysiert, inwiefern 
    die MVC-Implementierung zusätzlichen Codeaufwand erzeugt und 
    wie sich dies auf die Effizienz der Anwendung auswirkt.
\end{itemize}

Jede dieser Metriken dient dazu, die strukturellen Vorzüge sowie 
potenzielle Nachteile der MVC-Architektur zu ermitteln und eine 
objektive Analyse zu gewährleisten. Basierend auf diesen Kriterien 
werden die Ergebnisse in den folgenden Abschnitten diskutiert.